\documentclass{beamer}
\usetheme{Madrid}

\usepackage{graphics}
\usepackage{amsmath}
\usepackage{xcolor}
\usepackage{physics}

\title[Wobbling Motion in Odd-Mass Nuclei] % (optional, only for long titles)
{Wobbling Motion in odd-mass nuclei}
\author[R. Poenaru] % (optional, for multiple authors)
{Robert Poenaru\inst{1,2}}
\institute[DFT @ IFIN-HH] % (optional)
{
  \inst{1}%
  Department of Theoretical Physics\newline
  IFIN-HH
  \and
  \inst{2}%
  Faculty of Physics\newline
  University of Bucharest
}
\date[\today] % (optional)
{Bucharest University Faculty of Physics 2021 Meeting}
\subject{Nuclear Physics}

\begin{document}
\maketitle
\begin{frame}
\frametitle{Table of Contents}
\tableofcontents
\end{frame}

\section{Triaxial nuclei}

\begin{frame}{Triaxiality}
\begin{itemize}
    \item Nuclear shapes: most of the nuclei are spherical or axially symmetric in the ground state.
\end{itemize}
  \begin{figure}
    \centering
    % \includegraphics[scale=0.65]{figs/prolate.pdf}
    \caption{\textbf{Spherical:} $\beta_2=0$ ; \textbf{Prolate:} $\beta_2>0$ ; \textbf{Oblate:} $\beta_2<0$}
  \end{figure}
\end{frame}

\begin{frame}{Nuclear deformation}
    \begin{block}{Nuclear surface - axially-symmetric shape}
    \begin{itemize}
    \item The shape of the nucleus can be described via the \textit{deformation parameters} $\beta,\gamma$. 
    \item They arise from the shape parameterization of the nuclear surface.
    \begin{align}
        R(\theta,\phi)=R_0\left(1+\sum_{\lambda=2}^{\infty}\sum_{\mu=0}^{\lambda}\alpha_{\lambda\mu}Y_{\lambda\mu}(\theta,\phi)\right)
    \end{align}
    \item For axial-quadrupole deformed nuclei $\beta_2=\alpha_{20}$.
    \end{itemize}
    \end{block}
\end{frame}

\begin{frame}{Deformation parameters}
    There are also deviations from \emph{axial symmetric shapes} $\to$ \textbf{triaxial shapes} (e.g. no symmetry axis).
    \begin{block}{Nuclear surface - axially-asymmetric shape}
    \begin{itemize}
    \item For nuclei with the three principal axes of different lengths (\emph{axial asymmetry}), the $\gamma$ deformation parameter emerges.
    \item Relationship between $\beta,\gamma$ and the expansion coefficients from the expression of $R(\theta,\phi)$ is given by:
    \begin{align}
        \alpha_{20}&=\beta_2\cos\gamma\\
        \alpha_{22}&=\frac{1}{\sqrt{2}}\beta_2\sin\gamma
    \end{align}
    \item $\gamma$ is a measure of asymmetry between the three \textit{moments of inertia} of the nucleus.
    \end{itemize}
    \end{block}
\end{frame}

\begin{frame}
\frametitle{Shape parameters}
\begin{figure}
    \centering
    \includegraphics[scale=0.5]{figs/NuclearShapes1.png}
    \caption{The $(\beta,\gamma$) plane divided into six equivalent parts.}
    \label{fig:betagamma}
\end{figure}
\end{frame}

\begin{frame}{Wobbling motion  - clear signature for triaxiality}
  \begin{itemize}
    \item A triaxial nucleus can rotate about any of the three axes
    \item Rotation about the axis with the largest moment of inertia (MOI) is energetically the most favorable
    \item The other two axes contribute to the total nuclear motion (due to the anisotropy between the MOIs) $\to$ \color{red}{This motion has an oscillating behavior} 
  \end{itemize}
  \begin{block}{Wobbling motion (WM)}
  \begin{itemize}
    \item Uniquely associated to triaxial structures. 
    \item It was theoretically predicted by Bohr and Mottelson more than 50 years ago (for the even-$A$ case).
    \item Experimentally confirmed for $^{163}$Lu in 2001.
  \end{itemize}
  \end{block}
\end{frame}

\begin{frame}{Wobbling motion}

\begin{columns}
    \begin{column}{0.4\textwidth}
  \begin{figure}
    \centering
      \includegraphics[scale=0.5]{figs/simpleWobbler.pdf}
      \caption{A simple wobbler.}
  \end{figure}
  \end{column}
  \begin{column}{0.6\textwidth}
        \begin{figure}
          \centering
          \includegraphics[scale=0.35]{figs/wobbling_drawing.png}
          \caption{Schematic representation for the nuclear wobbling motion.}
          \label{wobbling_picture}
      \end{figure}
  \end{column}
  \end{columns}
\end{frame}


\begin{frame}{Wobbling motion}
  \begin{columns}
    \begin{column}{0.47\textwidth}
    \begin{block}{Triaxial nuclei}
    The rotational angular momentum is NOT aligned along any of the body-fixed axes: it \textbf{precesses} and \textbf{wobbles} around the axes with the largest MOI. 
    \end{block}
  \begin{block}{Wobbling bands}
  Sequences of $\Delta I=2\hbar$ rotational bands that are built on different \textit{wobbling phonon excitations}.
  \end{block}
    \end{column}
    \begin{column}{0.53\textwidth}  %%<--- here
   \begin{figure}
     \centering
     \includegraphics[scale=0.15]{figs/wobblingBands.png}
     \caption{Rotational-band structures of the wobbling motion.}
   \end{figure}
    \end{column}
    \end{columns}
\end{frame}


\begin{frame}{Particle-rotor coupling}
  \begin{block}{Wobbling motion in odd-A nuclei}  
    \begin{itemize}
      \item Coupling of a nucleon from a high j-shell with the triaxial rotor core is crucial for the wobbling phenomenon.
      \item The odd particle’s angular momentum couples to the rotor, driving the nucleus to large deformations and it also stabilizes the shape.
    \end{itemize}
  \end{block}
  For nuclei with $A\approx160$, the odd $\pi i_{13/2}$ is the \emph{intruder} which drive the isotope to very high deformations (up to $n_w=3$ wobbling phonon number).
  
  For nuclei with $A\approx180$, the odd $\pi h_{9/2}$ and $\pi i_{13/2}$ are the \emph{intruders} which drive the system to very high deformations.
\end{frame}


\begin{frame}{Wobbling regimes}
  Frauendorf et al (2014) formulated two possible wobbling modes in the case of $odd-A$ nuclei.
  \begin{block}{$\ $}
    \begin{description}
      \item[Longitudinal wobbling (WL)] the quasiparticle's angular momentum $j$ is \emph{aligned} with the rotational axis of the system ($m$-axis for triaxial nuclei)
      \item[Transverse wobbling (TW)] the quasiparticle's a.m. $j$ is \emph{perpendicular} to the rotational axis ($m$); so it is parallel with either the long ($l$) axes or the short ($s$ axes). 
    \end{description}
  \end{block}
\end{frame}

\begin{frame}{LW vs TW - graphical representation}
  \begin{columns}
    \begin{column}{0.5\textwidth}  %%<--- here
      \begin{figure}
        \centering
        \includegraphics[scale=0.6]{figs/LongitudinalWobbler.pdf}
      \end{figure}
    \end{column}
    \begin{column}{0.5\textwidth}
      \begin{figure}
        \centering
  \includegraphics[scale=0.6]{figs/transverseWobbler.pdf}

      \end{figure}
    \end{column}
    \end{columns}

\end{frame}

\section{Theoretical Formalism}

\begin{frame}{Theoretical framework}
  The odd-mass system consists of an {\color{red}even-even core} (described by a triaxial rotor Hamiltonian {\color{red}$H_\text{rot}$}) and a single {\color{blue}$j$-shell nucleon} described by its single-particle Hamiltonian {\color{blue}$H_\text{sp}$}.
  \par \textbf{Total system:}
  \begin{align}
    &H={\color{red}H_\text{rot}}+{\color{blue}H_\text{sp}}=\\
&{\color{red}\sum_{k=1,2,3}A_k(I_k-j_k)^2}+{\color{blue}\epsilon_j+\frac{V}{j(j+1)}\left[\cos\gamma(3j_3^2-\mathbf{j}^2)-\sqrt{3}\sin\gamma(j_1^2-j_2^2)\right]}
  \end{align}
  \par \textbf{Solving the Hamiltonian in semi-classical approach:} \textit{R. Poenaru and A. A. Raduta, International Journal of Modern Physics E, 2150033, 2021}.
\end{frame}

\begin{frame}{Energy function - Semiclassical approach}
      The eigenvalues for $\hat{H}=H_{rot}+H_{sp}$ are obtained by solving the \emph{variational principle}:
  \begin{align}
    \delta \int_0^t\bra{\Psi_{IjM}}\hat{H}-i\frac{\partial}{\partial t'}\ket{\Psi_{IjM}}dt'=0.
  \end{align}
  The trial function $\ket{\Psi_{IjM}}$ is a direct product $\ket{\psi}_\text{core}\otimes\ket{\phi}_\text{s.p.}$.
\end{frame}


\begin{frame}{Classical energy function}
Considering the \emph{energy function} as the average of the Hamiltonian on the trial function:
  \begin{align}
    \mathcal{H}\equiv\bra{\Psi_{IjM}}H\ket{\Psi_{IjM}}
  \end{align}
  \begin{block}{Dispersion equation}
  \begin{itemize}
      \item The energy function is \emph{minimal} - $\mathcal{H}_{I,\text{min}}$ - in the point $\mathcal{P}_{min}$ when $A_1<(A_2,A_3)$.
      \item Linearizing $\mathcal{H}$ around $\mathcal{P}$:
  \begin{align}
    \Omega^4+B\Omega^2+C=0
  \end{align}
  \item $B$ and $C$ being some expressions which depend on the inertia factors, single particle potential strength $V$ and the triaxiality parameter $\gamma$.
  \end{itemize}  
  \end{block}
\end{frame}

\begin{frame}{Semiclassical wobbling frequencies}
  Under some restrictions for the MOIs, the dispersion equation in variable $\Omega$ admits two real and positive solutions.\\
  \par \emph{Wobbling solutions:}
  \begin{align}
    (\Omega_1^I,\Omega_{1'}^I)\\
    (\Omega_2^I,\Omega_{2'}^I)
  \end{align}
  which are the solutions denoted for corresponding coupling of the rotor with $j=i_{13/2}$ and $j=h_{9/2}$, respectively.
  \par In our model(s), the pair of frequencies $({\color{blue}\Omega_k^I},{\color{red}\Omega_{k'}^I})$ is associated with the rotational motion of the {\color{blue}core} and the {\color{red}odd particle}, respectively.
\end{frame}

\begin{frame}{Semiclassical energies for the wobbling nucleus}
  %The excitation energies for a band is given in terms of the minimal energy function plus the \emph{zero-point} motion, accordingly:
  \begin{align}
    E_I^\text{TSD1}&={\color{cyan}\epsilon_{13/2}}+{\color{red}\mathcal{H_{I,\text{min}}}(13/2)}+{\color{blue}\frac{1}{2}\left(\Omega_1^I+\Omega_{1'}^I\right)},\ I=13/2, 17/2,\dots\\
    E_I^\text{TSD2}&={\color{cyan}\epsilon_{13/2}}+{\color{red}\mathcal{H_{I,\text{min}}}(13/2)}+{\color{blue}\frac{1}{2}\left(\Omega_1^I+\Omega_{1'}^I\right)},\ I=27/2, 31/2,\dots\\
    E_I^\text{TSD3}&={\color{cyan}\epsilon_{13/2}}+{\color{red}\mathcal{H_{I-1,\text{min}}}(13/2)}+{\color{magenta}\frac{1}{2}\left(3\Omega_1^{I-1}+\Omega_{1'}^{I-1}\right)},\ I=33/2, 37/2,\dots\\
    E_I^\text{TSD4}&={\color{cyan}\epsilon_{9/2}}+{\color{red}\mathcal{H_{I,\text{min}}}(9/2)}+{\color{blue}\frac{1}{2}\left(\Omega_2^I+\Omega_{2'}^I\right)},\ I=47/2, 51/2,\dots
  \end{align} 
  with each term representing: {\color{cyan}{single particle energy}}, the {\color{red}energy in the minimum point} and {\color{blue}zero-point energy}*.
  \par {\color{magenta} TSD3 has a one-phonon excitation built on top of TSD2.}
\end{frame}

\section{Results}  

\begin{frame}
    \frametitle{Example of columns 1}
    \begin{columns}[c] % the "c" option specifies center vertical alignment
    \column{.45\textwidth} % column designated by a command
     Contents of the first column
    \column{.45\textwidth}
     Contents split \newline into two lines
    \end{columns}
\end{frame}

\section{Conclusions}

  \begin{frame}
  \centering
    \Large{Thank you for your attention!}
  \end{frame}

\end{document}