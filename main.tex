\documentclass{beamer}
\usetheme{Madrid}

\title[Wobbling Motion in Odd-Mass Nuclei] % (optional, only for long titles)
{Wobbling Motion in odd-mass nuclei}
\author[R. Poenaru] % (optional, for multiple authors)
{Robert Poenaru\inst{1,2}}
\institute[DFT @ IFIN-HH] % (optional)
{
  \inst{1}%
  Department of Theoretical Physics\newline
  IFIN-HH
  \and
  \inst{2}%
  Faculty of Physics\newline
  University of Bucharest
}
\date[\today] % (optional)
{Bucharest University Faculty of Physics 2021 Meeting}
\subject{Nuclear Physics}

\begin{document}
\maketitle
\begin{frame}
\frametitle{Table of Contents}
\tableofcontents
\end{frame}

\section{Triaxial nuclei}

\begin{frame}{Triaxiality}
\begin{itemize}
    \item Nuclear shapes: most of the nuclei are spherical or axially symmetric in the ground state.
    \item There are also deviations from \emph{axial symmetric shapes} $\to$ \textbf{triaxial shapes} (e.g. no symmetry axis).
\end{itemize}
  \begin{figure}
    \centering
    % \includegraphics[scale=0.65]{figs/prolate.pdf}
    \caption{\textbf{Spherical:} $\beta_2=0$ ; \textbf{Prolate:} $\beta_2>0$ ; \textbf{Oblate:} $\beta_2<0$}
  \end{figure}
\end{frame}

\begin{frame}{Nuclear deformation}
    \begin{block}{Nuclear surface - axially-symmetric shape}
    \begin{itemize}
    \item The shape of the nucleus can be described via the \textit{deformation parameters} $\beta,\gamma$. 
    \item They arise from the shape parameterization of the nuclear surface.
    \begin{align}
        R(\theta,\phi)=R_0\left(1+\sum_{\lambda=2}^{\infty}\sum_{\mu=0}^{\lambda}\alpha_{\lambda\mu}Y_{\lambda\mu}(\theta,\phi)\right)
    \end{align}
    \item For axial-quadrupole deformed nuclei $\beta_2=\alpha_{20}$.
    \end{itemize}
    \end{block}
\end{frame}

\begin{frame}{Deformation parameters}
    \begin{block}{Nuclear surface - axially-asymmetric shape}
    \begin{itemize}
    \item For nuclei with the three principal axes of different lengths (\emph{axial asymmetry}), the $\gamma$ deformation parameter emerges.
    \item Relationship between $\beta,\gamma$ and the expansion coefficients from the expression of $R(\theta,\phi)$ is given by:
    \begin{align}
        \alpha_{20}&=\beta_2\cos\gamma\\
        \alpha_{22}&=\frac{1}{\sqrt{2}}\beta_2\sin\gamma
    \end{align}
    \item $\gamma$ is a measure of asymmetry between the three \textit{moments of inertia} of the nucleus.
    \end{itemize}
    \end{block}
\end{frame}

\begin{frame}
\frametitle{Shape parameters}
\begin{figure}
    \centering
    \includegraphics[scale=0.5]{figs/NuclearShapes1.png}
    \caption{The $(\beta,\gamma$) plane divided into six equivalent parts.}
    \label{fig:betagamma}
\end{figure}
\end{frame}

\begin{frame}{Wobbling motion}
  \begin{columns}
    \begin{column}{0.50\textwidth}
      The rotational angular momentum for triaxial nucleus is NOT aligned along any of the body-fixed axes: it \textbf{precesses} and \textbf{wobbles} around the axes with the largest MOI. 
  \begin{block}{Wobbling bands}
  Sequences of $\Delta I=2\hbar$ rotational bands that are built on different \textit{wobbling phonon excitations}.
  \end{block}
    \end{column}
    \begin{column}{0.50\textwidth}  %%<--- here
   \begin{figure}
     \centering
     \includegraphics[scale=0.15]{figs/wobblingBands.png}
     \caption{Rotational-band structures of the wobbling motion.}
   \end{figure}
    \end{column}
    \end{columns}
\end{frame}

\begin{frame}{Particle-rotor coupling}
  \begin{block}{Wobbling motion in odd-A nuclei}  
    \begin{itemize}
      \item Coupling of a nucleon from a high j-shell with the triaxial rotor core is crucial for the wobbling phenomenon.
      \item The odd particle’s angular momentum couples to the rotor, driving the nucleus to large deformations and it also stabilizes the shape.
    \end{itemize}
  \end{block}
  For nuclei with $A\approx160$, the odd $\pi i_{13/2}$ is the \emph{intruder} which drive the isotope to very high deformations (up to $n_w=3$ wobbling phonon number).
  
  For nuclei with $A\approx180$, the odd $\pi h_{9/2}$ and $\pi i_{13/2}$ are the \emph{intruders} which drive the system to very high deformations.
\end{frame}

\begin{frame}{Nuclear Wobbling Motion}
      \begin{figure}
          \centering
          \includegraphics[scale=0.45]{figs/wobbling_drawing.png}
          \caption{Schematic representation for the nuclear wobbling motion.}
          \label{wobbling_picture}
      \end{figure}
\end{frame}


\begin{frame}{Wobbling regimes}
  Frauendorf et al (2014) formulated two possible wobbling modes in the case of $odd-A$ nuclei.
  \begin{block}{$\ $}
    \begin{description}
      \item[Longitudinal wobbling (WL)] the quasiparticle's angular momentum $j$ is \emph{aligned} with the rotational axis of the system ($m$-axis for triaxial nuclei)
      \item[Transverse wobbling (TW)] the quasiparticle's a.m. $j$ is \emph{perpendicular} to the rotational axis ($m$); so it is parallel with either the long ($l$) axes or the short ($s$ axes). 
    \end{description}
  \end{block}
  \textbf{Important:} in the present case (the quasiparticle has a particle character), so it aligns its a.m. with the short ($s$) axes. (Sensharma et al 2020).
\end{frame}

\begin{frame}{LW vs TW - graphical representation}
  \begin{columns}
    \begin{column}{0.5\textwidth}  %%<--- here
      \begin{figure}
        \centering
        \includegraphics[scale=0.6]{figs/LongitudinalWobbler.pdf}
      \end{figure}
    \end{column}
    \begin{column}{0.5\textwidth}
      \begin{figure}
        \centering
  \includegraphics[scale=0.6]{figs/transverseWobbler.pdf}

      \end{figure}
    \end{column}
    \end{columns}

\end{frame}

\section{Results}  

\begin{frame}
    \frametitle{Example of columns 1}
    \begin{columns}[c] % the "c" option specifies center vertical alignment
    \column{.45\textwidth} % column designated by a command
     Contents of the first column
    \column{.45\textwidth}
     Contents split \newline into two lines
    \end{columns}
\end{frame}

\section{Conclusions}

  \begin{frame}
  \centering
    \Large{Thank you for your attention!}
  \end{frame}

\end{document}